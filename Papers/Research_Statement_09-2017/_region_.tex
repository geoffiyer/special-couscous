\message{ !name(Research_Statement.tex)}\documentclass{article}[11pt]
\usepackage{Geoff,graphicx,subcaption}

\title{\vspace{-3cm}TODO\vspace{-0.3cm}}
\author{Geoffrey Iyer} \date{}
\begin{document}

\message{ !name(Research_Statement.tex) !offset(-3) }

\maketitle

\section{Introduction}
\label{sec:intro}

With the increasing availability of data we often come upon multiple
datasets, derived from different sensors, that describe the same object or
phenomenon. We call the sensors \emph{modalities}, and because each
modality represents some new degrees of freedom, it is generally desirable
to use more modalities rather than fewer. For example, in the area of
speech recognition, researchers have found that integrating the audio data
with a video of the speaker results in a much more accurate classification
\cite{Potamianos03, sedighin:hal-01400542}. Similarly, in medicine, the
authors of \cite{Lei12} and \cite{Samadi2016} fuse the results of two
different types of brain imaging to create a final image with better
resolution than either of the originals. However, correctly processing a multimodal dataset is not a simple task \cite{lahat:hal-01062366}. The main difficulty in multimodaly lies in finding a way to coordinate information that is represented in different formats. In the current state-of-the-art, most multimodal algorithms apply only to one specific problem. 

Now start talking about representations as manifolds and geometry or topology stuff. This is a very abstract way to view the data and is certainly portable to many different problems. You have plenty of stuff to cite here too.

Now get into graph matching. Not sure how much to put here and how much to put in the method section. I guess we'll decide as we move along.

\section{The Method}
\label{sec:method}

\subsection{The basic problem}
Let $X = \{x_1,x_2,\ldots,x_N\}$, $Y = \{y_1,\ldots,y_N\}$ be undirected, weighted graphs, with corresponding weight matrices $W_X,W_Y$. For convenience of notation, we will assume $\abs{A} = \abs{Y}$. Later we will show how to extend to the general case. Our goal is to find a bijection $\rho: X\to Y$ that preserves the edge weights. 

We will show how to extend in a bit. We want to find bijection \[p: \{1,2,\ldots,N\} \to \{1,2,\ldots,N\}\] that preserves the weights as best as possible. But it's easier to represent with a matrix. $P$ will be $N\times N$ and will have exactly one $1$ in each column (and a lot of zeros). We want to minimize
\[\norm{PW_1P^T - W_2}^2_F.\]
Exact solution is too expensive. Can solve using Graph Laplacian trick from \cite{Umeyama1988,Knossow2009}.

\subsection{Some stuff about the graph laplacian probably}
Relax problem to
\[Q^* = \text{argmin}_{QQ^T=I}\norm{QW_1Q^T - W_2}^2_F.\]
Using orthogonal instead of permutation makes this solvable by standard linear algebra. Here's the answer. Let $L_1,L_2$ the Graph Laplcians corresponding to $W_1,W_2$ (TODO: Actually define this reasonably). $U_1,U_2$ the corresponding matrices of eigenvectors. Then $Q^* = U_1SU_2^T$. $S$ is a diagonal matrix with entries of $\pm 1$ to account for sign ambiguity in eigenvectors.

This $S$ is actually a pain, because it represents $2^N$ possibilities. Right now we're supervising this away by assuming we know one point worth of correspondence. There are some ideas on how to fix it unsupervised in \cite{Knossow2009}.

Recall from Graph Laplacian
\begin{align*}
  \text{columns of }U_i &\iff \text{ features} \\
  \text{rows of }U_i &\iff \text{ data points}.
\end{align*}
Match rows of $U_1$ to rows of $U_2$ by considering $U_1U_2^T$. \\~\\

$Q^*_{ij}$ gives the similarity between node $i$ of $G_1$ and node $j$ of $G_2$. \\~\\
Choose a permutation $p: \{1,2,\ldots,N\} \to \{1,2,\ldots,N\}$ via
\[\text{argmax}_{\text{permutations }p}\sum_{i=1}^N Q^*_{i,p(i)}.\]
Hungarian algorithm finds this in $O(N^3)$.

Benefits of Graph Matching
\begin{enumerate}
\item A precise number representing similarity between nodes gives us many options.
  \begin{itemize}
  \item Thresholding
  \item Many-to-many matching
  \item Hierarchical matching
  \end{itemize}
\item Easy extension to the case $\abs{G_1}\neq\abs{G_2}$.
\item Robust to many continuous deformations.
  \begin{itemize}
  \item scaling, shifts, rotations, etc.
  \end{itemize}
\end{enumerate}

\subsection{Coarsifying}

NEED TO WRITE THIS ENTIRE THING

\subsection{Nystrom}

I can copy paste almost all of this from the other thing

\section{Experiment}
\label{sec:experiment}

Say we have $N=6$ and calculated:
\[Q^* = \begin{pmatrix}
    -0.1629 &  -0.1711 &  -0.1703 &   0.3426 &   0.3717 &  -0.2100\\
    -0.1647 &  -0.1662 &  -0.1677 &   0.2966 &   0.3192 &  -0.1172\\
    -0.1660 &  -0.1653 &  -0.1657 &  -0.1477 &  -0.1861 &   0.8308\\
    -0.4579 &   0.6860 &   0.2665 &  -0.1787 &  -0.1480 &  -0.1678\\
    0.4939 &  -0.1039 &   0.1196 &  -0.6689 &   0.3080 &  -0.1486\\
    0.4577 &  -0.0795 &   0.1176 &   0.3561 &  -0.6647 &  -0.1872\\
  \end{pmatrix}\]
Then \[P^* = \begin{pmatrix}
    0&0&0&1&0&0\\
    0&0&0&0&1&0\\
    0&0&0&0&0&1\\
    0&1&0&0&0&0\\
    1&0&0&0&0&0\\
    0&0&1&0&0&0\\
  \end{pmatrix}
\]

I also have a LOT of pictures I can copy-paste into this section. The couple of straight-up matches. Then the change detection work.

\section{Future Work}
\label{sec:futurework}

Oh boy, lots of stuff. Runtime fixes are interesting to me. Then there's all the decisions to make! We have this method that does something (although it's hard to see what). I can think of a bunch of things to do with it.

\bibliographystyle{unsrt} \bibliography{../../BibTex/research}

\end{document}
\message{ !name(Research_Statement.tex) !offset(-115) }
