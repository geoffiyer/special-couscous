\documentclass[12pt]{article}
\usepackage{Geoff}

\begin{document}

\section*{Brief description of match quality}

We start with datasets
\begin{align*}
  X &= \left\{x_1, x_2,\ldots, x_n\right\} \\
  Y &= \left\{y_1, y_2,\ldots, y_n\right\}, \\
\end{align*}
both of size $n\times 3$, coregistered. $X$ represents the after-flood image, and $Y$ represents the before-flood. We match these sets using our graph algorithm. Let's represent this matching as
\[x_i \text{ matches to } y_{\phi(i)}.\]
There is also the trivial matching via coregistration, where $x_i$ matches to $y_i$. Following your advice on Thursday, I decided to calculate match quality via
\[\norm{x_i - x_{\phi(i)}}.\]
Here a large number represents a bad match, and a small number represents a good match. The graph in the picture shows all of the different match qualities, sorted so that it's easier to view.

\end{document}